\documentclass[runningheads]{llncs}
\usepackage{longtable}
\usepackage{nth}

\begin{document}
\frontmatter
\setcounter{page}{5}

\chapter*{Preface to the Proceedings of HS3 2025}
\markboth{Preface}{Preface}

This volume presents the revised papers of the 1st Workshop on
Hardware-Supported Software Security (HS3 2025), co-located with the 30th
European Symposium on Research in Computer Security (ESORICS 2025), held in
Toulouse, France, on the \nth{25} of September, 2025.

The HS3 workshop seeks to share experience, tools and methodology on
hardware-assisted software security. We are looking forward to submissions
that propose new architectures offering better resilience against software
attacks. These architectures should rely on hardware-based security
mechanisms to protect the software stack. One of the challenges is to
formally specify and verify the security guarantees offered by such
architectures and to better assess the security guarantees provided by
existing hardware architectures against software attacks, especially
attacks against micro-architecture. This can be achieved by identifying new
vulnerabilities using reverse engineering, fuzzing or other attack
approaches. The goal of the HS3 workshop is to provide a forum for
researchers and practitioners from academia, industry and government that
work on hardware-assisted software security. HS3 2025 has a special theme
on “Secure Monitoring, Intrusion Detection, and Incident Response” and we
are specifically looking forward to submissions that present
hardware-assisted approaches in this domain.

Combining software and hardware aspects to consider new software attacks is
becoming increasingly important. For example, hardware vulnerabilities such
as Spectre or Meltdown can be exploited purely by software attacks. Such
attacks can be executed remotely and do not require physical access to the
targeted hardware platform. On the other hand, hardware features can be
used to better detect and respond to traditional software attacks, such as
memory corruption. Therefore, it is necessary to study the security of
software/hardware interfaces in terms of attacks and defenses.

The purpose of the HS3 workshop is to share experience, tools and
methodology on hardware-assisted software security. On one hand, we need to
propose new architectures offering better resilience against software
attacks. These architectures should rely on hardware-based security
mechanisms to protect the software stack. One of the challenges is to
formally specify and verify the security guarantees offered by such
architectures. On the other hand, we also need to assess better the
security guarantees provided by existing hardware architectures against
software attacks, especially attacks against micro-architecture. This can
be achieved by identifying new vulnerabilities using reverse engineering,
fuzzing or other attack approaches. The goal of the HS3 workshop is to
provide a forum for researchers and practitioners from academia, industry
and government that work on hardware-assisted software security.

Special Theme: Secure Monitoring, Intrusion Detection, and Incident
Response

Intrusion Detection System have become ubiquitous cybersecurity tools that
monitor network traffic or other system activity to then identify anomalous
behavior or policy violations. Upon detection, these systems may alert
security teams or central security systems, reporting potential threats, or
even trigger automated responses. For this year’s edition of the HS3
workshop, we especially encourage submissions that investigate questions
regarding the hardware-supported design of such systems. We are
specifically interested in submissions in the area of secure monitoring,
intrusion detection, and incident response, and we seek to develop a
special track, potentially with invited talks or panel discussions, on this
domain.


The call for papers attracted signifi-
cant interest, resulting in 42 submissions, underscoring the workshop’s prominence in
advancing research in this field. Each manuscript underwent a rigorous peer-review pro-
cess, with 2–4 single-blinded reviewers per paper, and all chairs and committee members
refraining from reviewing their own submissions. Ultimately, we accepted 10 regular
papers and 5 short papers, achieving a 35% acceptance rate.

Special thanks to our invited speaker ...

Thanks to the programme committee and external reviewers, to all speakers
and participants for ...



The final Table of Contents is created by Springer from the title information
in the papers. However, we ask you to provide us with a preliminary Table of
Contents when you send the files. This should contain the titles of the
papers and the names of the authors in the order in which they are to appear
in the volume and should include topical section headings. The papers should
be grouped according to the topics they address and not according to the
sessions of the conference. Any session numbering will be deleted by our
typesetters. Please include all papers pertaining to one topic under one
single topical heading, even if there are/were two or three such sessions at
the conference. If the titles given in your preliminary Table of Contents
differ from those in the papers, then we take the paper titles to be the
correct ones and create the final Contents accordingly.

The Author Index is also generated at Springer, but you can help us present
the authors’ names in the correct way by submitting a list of authors who
have complex family names, particles, or suffixes, or do not use the Western
name order (i.e., given name(s); family name). It should be made quite clear,
which part of the name is the given name and which is the family name.

Please check that the names of the authors are written consistently
throughout the proceedings volume. If an author has contributed to more than
one paper, their name should be spelt and structured in an identical manner
in all papers as well as in the Table of Contents.

At the end of the preface, the month and year should be inserted on the
left-hand side and the names of the authors of the preface (i.e., the PC
Chairs) should be listed on the right-hand side. Please note that pages I--IV
(in front of the Preface) are prepared by Springer.
\medskip

\begin{flushright}
\noindent November 2025\hfill
{Guillaume Hiet} \\
{Yuko Hara} \\
{Jan Tobias M\"uhberg} \\
\end{flushright}


\chapter*{Organization of HS3 2025}
\markboth{Organization}{Organization}
\setlength{\tabcolsep}{0pt}     % remove spacing between columns in the following long tables
\setlength{\LTpre}{0pt}         % remove extra vertical space before the long tables

\section*{Program Committee Chairs}
\begin{longtable}{p{0.3\textwidth}p{0.7\textwidth}}
Guillaume Hiet       & IRISA, CentraleSup\'elec/Inria, France \\
Yuko Hara            & Institute of Science, Tokyo, Japan \\
Jan Tobias M\"uhberg & Universit\'e Libre de Bruxelles, Belgium \\
\end{longtable}

\section*{Program Committee}
\begin{longtable}{p{0.3\textwidth}p{0.7\textwidth}}
Iness Ben Guirat     & Universit\'e Libre de Bruxelles, Belgium \\
Pascal Cotret        & ENSTA Bretagne, France \\
Kevin Cheang         & University of California, Berkeley, USA  \\
Chris Dalton         & HP Labs, United Kingdom \\
Lesly-Ann Daniel     & KU Leuven, Belgium \\
Merve G\"ulmez       & Ericsson Research, Sweden \& KU Leuven, Belgium \\
Karine Heydemann     & Thales, France \\
Vianney Lap\^otre    & Universit\'e Bretagne Sud, France \\
Cl\'ementine Maurice & CNRS, France \\
Maria M\'endez Real  & Universit\'e Bretagne Sud, France \\
Cristofaro Mune      & Raelize B.V., Malta \\
Antonio Mu\~noz      & University of M\'alaga, Spain \\
Kaveh Razavi         & ETH Zurich, Switzerland \\
Simon Rokicki        & ENS Rennes, France \\
Shweta Shinde        & ETH Zurich, Switzerland \\
Volker Stolz         & HVL, Sweden \\
Marcus V\"olp        & University of  Luxembourg, Luxembourg \\
Pierre Wilke         & CentraleSup\'elec/Inria, France \\
\end{longtable}

\section*{Additional Reviewers}
\begin{multicols}{2}
\noindent Navid Ladner \\
\end{multicols}

\tableofcontents

\mainmatter

%Create dummy chapters for the sample ToC

\setcounter{page}{1}
\addtocontents{toc}{\protect\section*{Attacks \& Vulnerabilities}}
\author{Antoine Plin and Fr\'ed\'eric Fauberteau and Nga Nguyen}
\title{OpenGL GPU-Based Rowhammer Attack (Work in Progress)}
\maketitle
\clearpage

\setcounter{page}{2}
\author{Mohammed Mezaouli and Yehya Nasser and Samir Saoudi and Marc-Oliver Pahl}
\title{Revealing Embedded System Behaviors: A Comparative Analysis of Power Consumption and Hardware Performance Counters}
\maketitle
\clearpage

\setcounter{page}{3}
\author{Germany Is Rolling Out Nation-Scale Key Escrow And Nobody Is Talking About It}
\title{Jan Sebastian G\"otte}
\maketitle
\clearpage

\setcounter{page}{4}
\addtocontents{toc}{\protect\section*{Defenses \& Anomaly Detection}}
\author{Victor Breux and Pierre-Henri Thevenon}
\title{Hardware Performance Counters for Anomaly Detection in Embedded Devices}
\maketitle
\clearpage

\setcounter{page}{5}
\author{Qingyu Zeng and Songxuan Liu and Yu Wu and Yuko Hara}
\title{Semantic-Aware Provenance-Based Intrusion Detection for Edge Systems}
\maketitle
\clearpage

\setcounter{page}{6}
\author{Emiliia Geloczi and Stefan Katzenbeisser}
\title{Inter-Device PUFs: A Novel Paradigm for Physical Unclonable Functions}
\maketitle
\clearpage

\setcounter{page}{7}
\author{L\'eo Cosseron, Louis Rilling and Martin Quinson}
\title{Mitigation of the impact of Virtual Machine Introspection Pauses on Multi-core Virtual Machines}
\maketitle
\clearpage


\setcounter{page}{8}
\phantom{Author Index}
\addcontentsline{toc}{title}{\protect\textbf{Author Index}}
\clearpage

\end{document}

