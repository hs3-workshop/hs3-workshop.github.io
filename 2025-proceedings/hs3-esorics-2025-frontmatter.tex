\documentclass[runningheads]{llncs}
\usepackage{longtable}
\usepackage{nth}
\usepackage[english=american,babel=false]{csquotes}

\begin{document}
\frontmatter
\setcounter{page}{5}

\chapter*{Preface to the Proceedings of HS3 2025}
\markboth{Preface}{Preface}

This volume presents the revised papers of the 1st Workshop on
Hardware-Supported Software Security (HS3 2025), co-located with the 30th
European Symposium on Research in Computer Security (ESORICS 2025), held in
Toulouse, France, on the \nth{25} of September, 2025.

The HS3 workshop seeks to share experience, tools and methodology on
hardware-assisted software security. Combining software and hardware
aspects to consider new software attacks is becoming increasingly
important. For example, hardware vulnerabilities such as Spectre or
Meltdown can be exploited purely by software attacks. Such attacks can be
executed remotely and do not require physical access to the targeted
hardware platform. However, hardware features can be used to better detect
and respond to traditional software attacks, such as memory corruption. The
goal of the HS3 workshop was to provide a forum for researchers and
practitioners from academia, industry and government, who study the
security of software/hardware interfaces in terms of attacks and defenses,
and hardware-supported software security.

We were looking for submissions that propose new processor architectures
or processor extensions that offer better resilience against software
attacks. These architectures typically provide hardware-based security
mechanisms to protect the software stack.  One of the key challenges in the
field is to formally specify and verify the security guarantees offered by
such architectures and to better assess the security guarantees provided by
existing hardware architectures against software attacks, especially
attacks against micro-architecture. This can be achieved by identifying new
vulnerabilities using reverse engineering, fuzzing or other attack
approaches. HS3 2025 also had a special theme on \enquote{Secure
Monitoring, Intrusion Detection, and Incident Response} and we were
specifically inviting submissions that present hardware-assisted approaches
in this domain.

Given that HS3 is a very young workshop, running for the first time in
co-location with ESORICS, our call for papers attracted significant interest
and resulted in 14 submissions. Each manuscript underwent a rigorous
peer-review process, with 3 single-blinded reviewers per paper, and
guaranteeing that chairs and committee members would not review their own or
otherwise conflicting submissions. Ultimately, we accepted 3 regular papers
and 4 short papers, achieving a 50\,\% acceptance rate.

The workshop itself featured 10 presentations (including 3 work-in-progress
submissions that are not present in the proceedings), an invited talk, and
time for discussion and reflection. With over 30 participants, the workshop
was well frequented and lively discussions around recent trends in
hardware-supported security and application-related topics.

We mould like to thank our invited speaker, Dr. Robert Norton, for
delivering an inspiring keynote on \enquote{Hardware-software co-design for
security with CHERIoT: from memory-safety to software supply-chain
resilience}.  We further thank programme committee and external reviewers
for their tireless commitment during the advertising and reviewing phase.
Most importantly, we thank all our speakers and participants for the great
presentations and intriguing discussions during the workshop. Finally we
would like to thank the ESORICS organization and specifically the workshop
chairs for supporting us throughout the process of preparing and running
HS3: Next year you will have to allocate a bigger room for us.

\medskip

\begin{flushright}
\noindent November 2025\hfill
{Guillaume Hiet} \\
{Yuko Hara} \\
{Jan Tobias M\"uhberg} \\
\end{flushright}


\chapter*{Organization of HS3 2025}
\markboth{Organization}{Organization}
\setlength{\tabcolsep}{0pt}     % remove spacing between columns in the following long tables
\setlength{\LTpre}{0pt}         % remove extra vertical space before the long tables

\section*{Program Committee Chairs}
\begin{longtable}{p{0.3\textwidth}p{0.7\textwidth}}
Guillaume Hiet       & IRISA, CentraleSup\'elec/Inria, France \\
Yuko Hara            & Institute of Science, Tokyo, Japan \\
Jan Tobias M\"uhberg & Universit\'e Libre de Bruxelles, Belgium \\
\end{longtable}

\section*{Program Committee}
\begin{longtable}{p{0.3\textwidth}p{0.7\textwidth}}
Iness Ben Guirat     & Universit\'e Libre de Bruxelles, Belgium \\
Pascal Cotret        & ENSTA Bretagne, France \\
Kevin Cheang         & University of California, Berkeley, USA  \\
Chris Dalton         & HP Labs, United Kingdom \\
Lesly-Ann Daniel     & KU Leuven, Belgium \\
Merve G\"ulmez       & Ericsson Research, Sweden \& KU Leuven, Belgium \\
Karine Heydemann     & Thales, France \\
Vianney Lap\^otre    & Universit\'e Bretagne Sud, France \\
Cl\'ementine Maurice & CNRS, France \\
Maria M\'endez Real  & Universit\'e Bretagne Sud, France \\
Cristofaro Mune      & Raelize B.V., Malta \\
Antonio Mu\~noz      & University of M\'alaga, Spain \\
Kaveh Razavi         & ETH Zurich, Switzerland \\
Simon Rokicki        & ENS Rennes, France \\
Shweta Shinde        & ETH Zurich, Switzerland \\
Volker Stolz         & HVL, Sweden \\
Marcus V\"olp        & University of  Luxembourg, Luxembourg \\
Pierre Wilke         & CentraleSup\'elec/Inria, France \\
\end{longtable}

\section*{Additional Reviewers}
\begin{multicols}{2}
\noindent Navid Ladner \\
\end{multicols}

\tableofcontents

\mainmatter

%Create dummy chapters for the sample ToC

\setcounter{page}{1}
\addtocontents{toc}{\protect\section*{Attacks \& Vulnerabilities}}
\author{Antoine Plin and Fr\'ed\'eric Fauberteau and Nga Nguyen}
\title{OpenGL GPU-Based Rowhammer Attack (Work in Progress)}
\maketitle
\clearpage

\setcounter{page}{2}
\author{Mohammed Mezaouli and Yehya Nasser and Samir Saoudi and Marc-Oliver Pahl}
\title{Revealing Embedded System Behaviors: A Comparative Analysis of Power Consumption and Hardware Performance Counters}
\maketitle
\clearpage

\setcounter{page}{3}
\author{Germany Is Rolling Out Nation-Scale Key Escrow And Nobody Is Talking About It}
\title{Jan Sebastian G\"otte}
\maketitle
\clearpage

\setcounter{page}{4}
\addtocontents{toc}{\protect\section*{Defenses \& Anomaly Detection}}
\author{Victor Breux and Pierre-Henri Thevenon}
\title{Hardware Performance Counters for Anomaly Detection in Embedded Devices}
\maketitle
\clearpage

\setcounter{page}{5}
\author{Qingyu Zeng and Songxuan Liu and Yu Wu and Yuko Hara}
\title{Semantic-Aware Provenance-Based Intrusion Detection for Edge Systems}
\maketitle
\clearpage

\setcounter{page}{6}
\author{Emiliia Geloczi and Stefan Katzenbeisser}
\title{Inter-Device PUFs: A Novel Paradigm for Physical Unclonable Functions}
\maketitle
\clearpage

\setcounter{page}{7}
\author{L\'eo Cosseron, Louis Rilling and Martin Quinson}
\title{Mitigation of the impact of Virtual Machine Introspection Pauses on Multi-core Virtual Machines}
\maketitle
\clearpage


\setcounter{page}{8}
\phantom{Author Index}
\addcontentsline{toc}{title}{\protect\textbf{Author Index}}
\clearpage

\end{document}

